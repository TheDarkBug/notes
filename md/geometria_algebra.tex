% to compile: pdflatex -synctex=1 -interaction=nonstopmode -file-line-error -recorder geometria_algebra.tex
\documentclass{article}
\usepackage{hyperref}
\usepackage{color}
\usepackage{amsfonts}
\usepackage{amsmath}
\usepackage[paperheight=1000em,margin=3em]{geometry}

\pagenumbering{gobble}
\title{Appunti del corso di algebra e geometria}
\author{Adriano Oliviero}
\date{07/04/22 - oggi}

\begin{document}
\newcommand{\hl}[1]{\colorbox{yellow}{#1}}
\newcommand{\ul}[1]{\underline{#1}}
\newcommand{\Def}[2]{\paragraph{\ul{Def}:}#1\\\hspace*{3em}\begin{minipage}{.8\textwidth}#2\end{minipage}}
\newcommand{\Esempio}[1]{\subsubsection*{\ul{Esempio}:}#1}
\newcommand{\R}{\mathbb{R}}
\maketitle
\section*{\hl{Introduzione}}
Questi appunti sono scritti seguendo le lezioni della prof.ssa \href{https://aiezzi.it/}{Anna Iezzi} e sono distribuiti sotto la \hyperref[sec:Licenza]{licenza}.\\
Il repository originario per questi (ed altri appunti) è su \href{https://github.com/TheDarkBug/notes}{github}.

\section*{\hl{Licenza}}
\label{sec:Licenza}
TO BE FILLED
\section{\hl{08/03/22}}
\subsection*{Richiami di logica}
Proposizione logica: \textbf{V}/\textbf{F}, \textbf{1}/\textbf{0} valore di verità
P = "Napoli è in Campania": \textbf{V}
P(n) = "n è pari"
P(2): \textbf{V}
P(3): \textbf{F}
\subsection*{Connettivi logici}
\begin{itemize}
	\item Negazione: $\neg$, "non"
	\item Congiunzione: $\wedge$, "e"
	\item Disgiunzione: $\vee$, "o"
	\item Implicazione: $\Rightarrow$, "se $\cdots$ allora $\cdots$"
	\item Doppia implicazione: $\Leftrightarrow$, "se e solo se"
\end{itemize}
\subsection*{Tavole di verità}
\begin{displaymath}
	\begin{array}{|c|c|c|c|c|c|c|c|c|c|c|}
		\hline
		P & Q & \neg P & P\wedge Q & P\vee Q & P\Rightarrow Q & P\Leftrightarrow Q \\
		\hline\hline
		1 & 1 & 0      & 1         & 1       & 1              & 1                  \\
		\hline
		1 & 0 & 0      & 0         & 1       & 0              & 0                  \\
		\hline
		0 & 1 & 1      & 0         & 1       & 1              & 0                  \\
		\hline
		0 & 0 & 1      & 0         & 0       & 1              & 1                  \\
		\hline
	\end{array}
\end{displaymath}
\subsection*{Quantificatori}
\begin{itemize}
	\item $\forall$: "Per ogni"
	\item $\exists$: "Esiste almeno uno"
	\item $\exists!$: "Esiste ed è unico"
\end{itemize}

\ul{Esempi}:
\begin{enumerate}
	\item "$\forall n$ naturale, $n$ è pari": F\\
	      "$\exists n$ naturale: ($\leftarrow$ tale che) $n$ è pari": V
	\item $P(x)=\{x$ è uno studente in aula A3-T2$\}$\\
	      $Q(x)=\{x$ è iscritto ad un corso di ingegneria$\}$\\
	      $\forall x, P(x)\Rightarrow Q(x)$
	\item $P(n)=\{n$ è un numero pari$\}$\\
	      $Q(n)=\{n$ è divisibile per $4\}$\\
	      $\forall n$ naturale, $Q(n)\Rightarrow P(n)$
\end{enumerate}

\subsection*{Teoria degli insiemi}
Un \textbf{insieme} è una collezione di oggetti detti \textbf{elementi} dell'insieme.\\
Convenzionalmente gli insiemi si denotano con lettere maiuscole e gli elementi con lettere minuscole.
\subsubsection*{Descrizione di un insieme}
\begin{enumerate}
	\item Per elencazione (se l'insieme ha un numero finito di elementi):\\
	      $\mathbb{A}=\{0,2,4,6,8,10\}$\\
	      $4\in\mathbb{A}$\\
	      $5\notin\mathbb{A}$
	\item Per proprietà caratteristica:\\
	      $\mathbb{A}=\{n:n$ è un numero pari: $0\le n\le10\}$
	\item Diagramma di \textit{Eulero-Venn} (ancora una volta se l'insieme ha un numero finito di elementi):
\end{enumerate}
\subsubsection*{Alcune informazioni sugli insiemi}
\begin{itemize}
	\item La \textbf{cardinalità} di un insieme è il numero di elementi che un insieme contiene e si denota: $\mathbb{A}$ ha cardinalità $|\mathbb{A}|$
	\item Insieme vuoto: insieme che non contiene nessun elemento, si denota con $\emptyset$, $\{\}$ e ha cardinalità $|\emptyset|=0$
	\item I principali insiemi numerici:\\
	      $\mathbb{N}=\{0,1,2,3,\cdots\}$: numeri naturali\\
	      $\mathbb{Z}=\{\cdots,-3,-2,-1,0,1,2,3,\cdots\}$: numeri naturali\\
	      $\mathbb{Q}=\{\frac{a}{b}:a,b$ e $\mathbb{Z},b\not=0\}$\\
	      $\R$: numeri reali\\
	      $\mathbb{C}$: numeri complessi
\end{itemize}

\subsubsection*{Operazioni tra gli insiemi}
\begin{enumerate}
	\item Inclusione $\leftrightarrow\subset$:\\
	      $\mathbb{A}\subseteq\mathbb{B}\quad(\mathbb{A}$ è contenuto in $\mathbb{B})$\\
	      $\mathbb{B}\supseteq\mathbb{A}$ $(\mathbb{B}$ contiene $\mathbb{A})$
	      \Esempio{$\mathbb{N}\subseteq\mathbb{Z}\subseteq\mathbb{Q}\subseteq\R\subseteq\mathbb{C}$}\\
	      \ul{Proprietà}:
	      \begin{itemize}
		      \item $\mathbb{A}=\mathbb{B}\Leftrightarrow(x\in\mathbb{A}\Leftrightarrow x\in\mathbb{B})\Leftrightarrow(\mathbb{A}\subseteq\mathbb{B}\wedge\mathbb{B}\subseteq\mathbb{A})$
	      \end{itemize}
	\item Intersezione $\leftrightarrow\wedge$:\\
	      $\mathbb{A}\cap\mathbb{B}=\{x:x\in\mathbb{A}\wedge x\in\mathbb{B}\}$\\
	      \ul{Proprietà}:
	      \begin{itemize}
		      \item $\mathbb{A}\cap\emptyset=\emptyset$
		      \item $\mathbb{A}\cap\mathbb{B}=\mathbb{B}\cap\mathbb{A}$
		      \item $\mathbb{A}\subseteq\mathbb{B}\Rightarrow\mathbb{A}\cap\mathbb{B}=\mathbb{A}$
	      \end{itemize}
	\item Unione $\leftrightarrow\vee$:\\
	      $\mathbb{A}\cup\mathbb{B}=\{x:x\in\mathbb{A}\vee x\in\mathbb{B}\}$\\
	      \ul{Proprietà}:
	      \begin{itemize}
		      \item $\mathbb{A}\cup\emptyset=\mathbb{A}$
		      \item $\mathbb{A}\cup\mathbb{B}=\mathbb{B}\cup\mathbb{A}$
		      \item $\mathbb{A}\subseteq\mathbb{B}\Rightarrow\mathbb{A}\cup\mathbb{B}=\mathbb{B}$
		      \item $\mathbb{A}\subseteq\mathbb{A}\cup\mathbb{B},\mathbb{B}\subseteq\mathbb{A}\cup\mathbb{B}$
	      \end{itemize}
	\item Differenza $\leftrightarrow\backslash$:\\
	      $\mathbb{B}\backslash\mathbb{A}=\{x:x\in\mathbb{B}\wedge x\not\in\mathbb{A}\}$
	\item Prodotto cartesiano $\leftrightarrow\times$:\\
	      $\mathbb{A}\times\mathbb{B}:=\{(a,b):a\in\mathbb{A},b\in\mathbb{B}\}$ con $(a,b)$ coppie ordinate\\
	      \ul{Proprietà}:
	      \begin{itemize}
		      \item $|\mathbb{A}\times\mathbb{B}|=|\mathbb{A}|\cdot|\mathbb{B}|$
		      \item $\mathbb{A}\times\emptyset=\emptyset=\emptyset\times\mathbb{A}$
	      \end{itemize}
\end{enumerate}
\subsection*{Funzioni}
\Def{Siano $\mathbb{A},\mathbb{B}$ due insiemi:}
{Una funzione $f:\mathbb{A}\rightarrow\mathbb{B}$ è una legge che associa ad ogni elemento di $\mathbb{A}$, uno ed un solo elemento di $\mathbb{B}$.\\
	$f:\mathbb{A}\rightarrow\mathbb{B}$\\
	\hspace*{1.6em}$x\rightarrow y:\quad x$ è la controimmagine di $y$}
\Def{Sia $f:\mathbb{A}\rightarrow\mathbb{B}$ una funzione e sia $\mathbb{X}\subseteq\mathbb{A}$:}
{$f(x):=\{f(x):x\in\mathbb{X}\}$ è l'insieme di $\mathbb{X}$ tramite $f$2
	$Im(f):=f(\mathbb{A})$ è l'insieme della funzione

	$Im(f)$ non è necessariamente $=Dom(f)$

	$Im(f)=Dom(f)\Rightarrow f$ è suriettiva

	$f(1)=f(2)=a\Rightarrow f$ non è iniettiva}
\section{\hl{09/03/22}}
\subsection*{Funzioni (continuo)}
\Def{Una funzione $f:\mathbb{A}\rightarrow\mathbb{B}$ si dice \ul{iniettiva} se $\forall x\not=y\Rightarrow f(x)\not=f(y)$}{(elementi distinti di $\mathbb{A}$ hanno immagini distinte) $\Leftrightarrow f(x)=f(y)\Rightarrow x=y$}
\Def{Una funzione $f:\mathbb{A}\rightarrow\mathbb{B}$ si dice \ul{suriettiva} se $Im(f)=B$,}{o equivalentemente se $\forall y\in\mathbb{B},\exists x\in\mathbb{A}:f(x)=y$}
\Def{Una funzione $f:\mathbb{A}\rightarrow\mathbb{B}$ si dice \ul{biettiva} o \ul{biunivoca}}{se è al tempo stesso iniettiva e suriettiva}
\Esempio{$f:\R\rightarrow\R\\\hspace*{1.6em}x\rightarrow x^2$
	\begin{itemize}
		\item È iniettiva? No, perchè $f(-1)=1=f(1)$
		\item È suriettiva? No, perchè $f(x)=x^2\geq0\forall x\in\R\Rightarrow Im(f)\subseteq\R^+\Rightarrow Im(f)\not=\R$
	\end{itemize}}
\vspace*{1em}
\par Se un insieme possiede delle operazioni che verificano certe proprietà, è una struttura algebrica.
\Def{Sia $\mathbb{X}$ un insieme, un'\ul{operazione binaria interna} è una funzione}{dal prodotto cartesiano $x\times x$ in $\mathbb{X}$.\\$*:x\times x \rightarrow x$\\\hspace*{1.4em}$(x,y)\rightarrow x*y$}
\subsubsection*{$\mathbb{X}=\R$}
$+:\R\times\R\rightarrow\R$\\
\hspace*{0em}$\qquad(x,y)\rightarrow x+y$

\ul{Proprietà} di $(\R,+)$:
\begin{enumerate}
	\item Commutatività: $x+y=y+x,\forall x,y\in\R$
	\item Associatività: $(x+y)+z=x+(y+z),\forall x,y,z\in\R$
	\item Elemento neutro: $\exists x'\in\R:x+x'=x'+x=x,\forall x\in\R,x'=0$
	\item Opposto: $\exists x'\in\R:x+x'=x'+x=0,\forall x\in\R,(x'=-x)$
\end{enumerate}
\par$\cdot:\R\times\R\rightarrow\R$\\
\hspace*{1.5em}$(x,y)\rightarrow x\cdot y$

\ul{Proprietà} di $(\R,\cdot)$:
\begin{enumerate}
	\item Commutatività: $x\cdot y=y\cdot x,\forall x,y\in\R$
	\item Associatività: $(x\cdot y)\cdot z=x\cdot(y\cdot z),\forall x,y,z\in\R$
	\item Elemento neutro: $\exists x'\in\R:x\cdot x'=x'\cdot x=x,\forall x\in\R,x'=1$
	\item Elemento inverso: $\exists x'\in\R:x\cdot x'=x'\cdot x=0,\forall x\in\R,(x'=\frac{1}{x})$
\end{enumerate}
Infine $+$ e $\cdot$ soddisfano la proprietà distributiva: $\forall x,y,z\in\R,x\cdot(y+z)=x\cdot y+x\cdot z$

Sia $\mathbb{K}\not=\emptyset$ un insieme dotato di due operazioni binarie:
\begin{itemize}
	\item $+:\mathbb{K}\times\mathbb{K}\rightarrow\mathbb{K}$
	\item $\cdot:\mathbb{K}\times\mathbb{K}\rightarrow\mathbb{K}$
\end{itemize}
$(\mathbb{K},+,\cdot)$ è detto un \ul{campo} se verificano le proprietà elencate prima, con $+$ distributiva e $\mathbb{K}$ al posto di $\R$

\Esempio{$\mathbb{F}_2=\{0, 1\}$
	\begin{itemize}
		\item $\cdot:\mathbb{F}_2\cdot\mathbb{F}_2\rightarrow\mathbb{F}_2$\\
		      \hspace*{1.8em}$(0,0)\rightarrow 0$\\
		      \hspace*{1.8em}$(0,1)\rightarrow 0$\\
		      \hspace*{1.8em}$(1,0)\rightarrow 0$\\
		      \hspace*{1.8em}$(1,1)\rightarrow 1$
		\item $+:\mathbb{F}_2\times\mathbb{F}_2\rightarrow\mathbb{F}_2$\\
		      \hspace*{1.8em}$(0,0)\rightarrow 0$\\
		      \hspace*{1.8em}$(0,1)\rightarrow 1$\\
		      \hspace*{1.8em}$(1,0)\rightarrow 1$\\
		      \hspace*{1.8em}$(1,1)\rightarrow 1$
	\end{itemize}}
\subsection*{Algebra lineare}
Wikipedia:
\begin{quotation}
	Branca della matematica che si occupa dello studio di spazi vettoriali (o anche detti spazi lineari), di trasformazioni lineari e di sistemi di equazioni lineari.
\end{quotation}
Molti problemi di matematica e fisica verificano la seguente proprietà:\\
Se $v,w\in\mathbb{X}$ sono due soluzioni del problema, allora anche $v+w$ e $\lambda v,\lambda\in\R$  ($+$ e $\cdot$ operazioni su $\mathbb{X}$) sono soluzioni del problema.\\
Problemi di questo tipo sono detti \textit{lineari}.

\subsubsection*{Nozione base: \ul{spazio vettoriale}}

I vettori sono usati in fisica per rappresentare, grandezze fisiche caratterizzate da:
\begin{itemize}
	\item una direzione
	\item un verso
	\item un'intensità
\end{itemize}
Tali grandezze sono \textbf{grandezze vettoriali} e si differenziano dalle \textbf{grandezze scalari}, definite unicamente dall'intensità.\\
Geometricamente, un vettore si rappresenta tramite un \textbf{segmento orientato} nel piano euclideo, denotato $\Pi$.
\Def{Un segmento orientato è una coppia di punti ordinata $(A,B\in\Pi\times\Pi)$.}{$\vec{AB}:=(A,B)$}
\Def{Due segmenti orientati $\vec{AB}$ e $\vec{CD}$ si dicono \ul{equipollenti} se il quadrilatero avente vertici, ordinatamente,}{$ABDC$ è un parallelogramma, quindi se hanno:
	\begin{itemize}
		\item stessa lunghezza
		\item direzione parallela
		\item stesso verso
	\end{itemize}}

\vspace*{2em}
L'equipollenza è una relazione di equivalenza, e verifica 3 proprietà:
\begin{itemize}
	\item riflessiva
	\item simmetrica
	\item transitiva
\end{itemize}
\Def{Un vettore geometrico è una classe di equipollenza}
{Sia $O$ un punto fissato nel piano $\Pi$, per ogni segmento orientato $\vec{AB}$ esiste un punto $P\in\Pi$ tale che $\vec{OP}$ è equipollente ad $\vec{AB}$.\\
	$\vec{OP}$ è equipollente a tutti i segmenti orientati equipollenti ad $\vec{AB}$ e posso sceglierlo come \ul{rappresentante} della classe di equipollenza di $\vec{AB}$.\\
	Quindi abbiamo una biezione:\\
	$\mathbb{V}=\{$Vettori geometrici nel piano$\}\{$segmento orientato $\vec{OP},P\in\Pi\}$\\
	Classi di equipollenza $\uparrow\hspace*{10em}\uparrow$ Stesso punto di applicazione}
\section{\hl{09/03/22} (data errata)}
\subsection*{Algebra lineare (continuo)}
\subsubsection*{Nozione base: \ul{spazio vettoriale} (continuo)}
Correzione della definizione di campo:
\begin{itemize}
	\item Esistenza dell'elemento neutro\\$\exists0\in\R:x+0=0+x=x,\forall x\in\R$
	\item Esistenza dell'opposto:\\$\exists x'\in\R:x+x'=x'+x=0,\forall x\in\R$
\end{itemize}
\hrulefill

\vspace*{1em}
Fissiamo $O\in\pi$ (piano ordinario).\\
$\mathbb{V}=\{$vettori geometrici del piano$\}\Leftrightarrow\{$segmenti orientati $\vec{OP},P\in\pi\}$\\
Con un abuso di notazione, consideriamo:\\
$\mathbb{V}=\{$segmenti orientati $\vec{OP},P\in\pi\}$\\
$\forall\vec{v}\in V,\exists P\in\pi:\vec{v}=\vec{OP}$
\ul{Operazioni su $\mathbb{V}$}
\begin{itemize}
	\item Somma di vettori\\
	      siano $\vec{v},\vec{w}\in \mathbb{V}$ e siano $P,Q\in\pi:$\\
	      $\vec{v}=\vec{OP}$\\
	      $\vec{w}=\vec{OQ}$\vspace*{1em}\\
	      Definiamo:\\
	      $\vec{v}+\vec{w}=\vec{OR}$, tale che il quadrilatero $OPRQ$ è un parallelogramma. (regola del parallelogramma)\\
	      \ul{Nota}: Se $O,P$ e $Q$ sono collineari (hanno la stessa direzione), costruisco $R$ tale che $OQ$ e $RP$ hanno la stessa lunghezza\\
	      Operazione binaria interna:\\
	      $+:\mathbb{V}\times\mathbb{V}\rightarrow\mathbb{V}$\\
	      \hspace*{1.9em}$(\vec{v},\vec{w})\rightarrow\vec{v}+\vec{w}$
	\item Moltiplicazione per scalari\\
	      Sia $\vec{v}\in\mathbb{V}$ e sia $P\in\pi:\vec{v}=\vec{OP}$.\vspace*{1em}\\
	      Definiamo $\lambda\cdot\vec{v}=\vec{OR}$
	      \begin{itemize}
		      \item $O,P,R$ sono collineari
		      \item $\overline{OR}=|\lambda|OP$
		      \item $\vec{OR}$ è orientato concordemente a $\vec{OP}\\\vec{OR}$ è orientato discordemente a $\vec{OP}$ se\\
		            $\lambda<0\vee\lambda=0,R=O$\\
		            Questa è un'operazione binaria \ul{esterna}:\\
		            $\cdot:\R\times\mathbb{V}\rightarrow \mathbb{V}$\\
		            \hspace*{1.5em}$(\lambda,\vec{v})\rightarrow\lambda\cdot\vec{v}$
	      \end{itemize}
\end{itemize}
\subsection*{Piano Cartesiano}
\par$\{P:P\in\pi\}\leftrightarrow\{(x,y):x,y\in\R\}=\R^2$\\
$\mathbb{V}=\{$segmenti orientati $\vec{OP},P\in\pi\}=\{$vettori geometrici nel piano$\}$\\
In particolare esiste una biezione:\\
$\mathbb{V}\leftrightarrow\R^2$\\
$\forall P\in\pi,\vec{OP}\rightarrow(x,y)$, dove $x,y$ sono rispettivamente ascissa e ordinata di $P$\\
$\vec{OP}\leftarrow(x,y)$ dove $P(x,y)$\\
Vogliamo tradurre le operazioni su $\mathbb{V}$ in operazioni su $\R^2$:
\begin{enumerate}
	\item $\vec{v}=\vec{OP},P(x_P,y_P)$\\
	      $\vec{w}=\vec{OQ},Q(x_Q,y_Q)$
	      $\vec{v}+\vec{w}=\vec{OR}:$ quali sono le coordinate di $R$?\\
	      $OPQR\Rightarrow A(x_A,y_A)$ parallelogramma con punto medio di $PQ$ e $OR$\\
	      $A$ punto medio di $PQ\Rightarrow\{x_A=\frac{x_P+x_Q}{2},y_A=\frac{y_P+x_Q}{2}\}$\\
	      $A$ punto medio di $OR\Rightarrow\{x_A=\frac{x_R}{2},y_A=\frac{y_R}{2}\}$\\
	      $\Rightarrow\{x_R=x_P+x_Q,y_R=y_P+y_Q\}$\\
	      Operazione binaria \ul{interna}:\\
	      $+:\R^2\times\R^2\rightarrow\R^2$\\
	      $((x_1,y_1),(x_2,y_2))\mapsto(x_1+x_2,y_1+y_2)$
	\item $\vec{v}\in\mathbb{V},\vec{v}=\vec{OP},P(x_P,y_P),\lambda\in\R$\\
	      $\lambda\cdot\vec{v}=\vec{OR}:$ quali sono le coordinate di $R$?\\
	      $\vec{OR}=\lambda\cdot\vec{OP}$\\
	      $OPH$ e $ORK$ sono simili per costruzione con rapporto di proporzioni $|\lambda|$.\\
	      \ul{Due casi}:
	      \begin{enumerate}
		      \item $\lambda\ge0$ $\vec{OR}$ è concord con $\vec{OP}$ e quindi:\\
		            $x_R=|\lambda|x_P=\lambda x_P$\\
		            $y_R=|\lambda|y_P=\lambda y_P$
		      \item $\lambda<0$ $\vec{OR}$ è discorde con $\vec{OP}$ e quindi:\\
		            $x_R=-|\lambda|x_P=\lambda x_P$\\
		            $y_R=-|\lambda|y_P=\lambda y_P$
	      \end{enumerate}
	      Operazione binaria \ul{esterna}:\\
	      $\cdot:\R\times\R^2\rightarrow\R^2$\\
	      \hspace*{0.1em}$(y,(x,y))\mapsto\lambda\cdot(x,y):=(\lambda x,\lambda y)$
\end{enumerate}
In conclusione abbiamo definito due operazioni su $\R^2$ "compatibili" con le operazioni definite su $\mathbb{V}$:
\begin{itemize}
	\item $+: \R^2\times\R^2\rightarrow\R^2$\\$((x_1,y_1),(x_2,y_2))\mapsto(x_1,y_1)+(x_2,y_2):=(x_1+y_1,x_2+y_2)$
	\item $\cdot: \R^2\times\R^2\rightarrow\R^2$\\\hspace*{-0.3em}$(\lambda,(x_2,y_2))\mapsto\lambda\cdot(x_2,y_2):=(\lambda\cdot x,\lambda\cdot y)$
\end{itemize}
\ul{Proprietà} di $+$ e $\cdot$ :
\begin{enumerate}
	\item Commutatività:\\$\forall(x_1,y_1),(x_2,y_2)\in\R^2, (x_1,y_1)+(x_2,y_2)=(x_2,y_2)+(x_1,y_1)$
	\item Associatività:\\$\forall(x_1,y_1),(x_2,y_2),(x_3,y_3)\in\R^2,\\((x_1,y_1)+(x_2,y_2))+(x_3,y_3)=(x_1,y_1)+((x_2,y_2)+(x_3,y_3))$
	\item Elemento neutro $(+)$:\\$(0,0)\in\R^2$ è tale che $(x,y)+(0,0)=(0,0)+(x,y)=(x,y)$
	\item Elemento opposto:\\$\forall(x,y)\in\R^2,\exists(x',y')\in\R^2:(x,y)+(x',y')=(x',y')+(x,y)=(0,0)$
	\item Distributività rispetto a vettori:\\$\forall(x_1,y_1),(x_2,y_2)\in\R^2,\forall\lambda\in\R$\\$\lambda\cdot((x_1,y_1)+(x_2,y_2))=\lambda\cdot(x_1,y_1)+\lambda\cdot(x_2,y_2)$
	\item Distributività rispetto a scalari:\\$\forall(x,y)\in\R^2,\forall\lambda,\mu\in\R$\\$(\lambda+\mu)\cdot(x,y)=\lambda\cdot(x,y)+\mu\cdot(x,y)$
	\item Senza nome:\\$\lambda\mu\cdot(x,y)=\lambda\cdot(\mu\cdot(x,y))$
	\item Elemento neutro $(\cdot)$:\\$1\cdot(x,y)=(x,y)\forall(x,y)\in\R^2$
\end{enumerate}
\subsection*{\hl{Revisione conclusa qui (29/03/22 19:42:57)}}
$(\R^2,+,\cdot)$ è il primo insieme di \hl{INCOMPLETO}
\Def{Sia $\mathbb{K}$ (K da \textit{korper} in tedesco) un campo. Uno spazio vettoriale su $\mathbb{K}$ è un insieme $\mathbb{V}$ dotato di due operazioni:}{
	\begin{itemize}
		\item $+: V\times V\rightarrow V$
		\item $\cdot: K\times V\rightarrow V$
	\end{itemize}
	Che verificano le seguenti proprietà:
	\begin{enumerate}
		\item Commutatività: $\forall v,w\in V, v+w=w+v$
		\item Associatività: $\forall u,v,w\in V,(u+v)+w=u+(v+w)$
		\item Elemento neutro:\\$\exists0\in V:0+v=v+0=v,\forall v\in V$
		\item Elemento opposto:\\$\forall v\in V,\exists v'\in V:v+v'=v'+v=0$
		\item Distributività rispetto alla somma di vettori:\\$\forall v,w\in V,\forall\lambda\in K,\lambda\cdot(v+w)=\lambda\cdot v+\lambda\cdot w$
		\item Distributività rispetto alla somma di scalari:\\$\forall v\in V,\forall\lambda,\mu\in K,(\lambda+\mu)\cdot v=\lambda\cdot v+\mu\cdot v$
		\item $\forall v\in V,\forall\lambda,\mu\in K,\lambda\mu\cdot v=\lambda\cdot(\mu\cdot v)$
		\item $1\cdot v=v,\forall v\in V$
	\end{enumerate}
	Gli elementi di $\mathbb{V}$ sono chiamati vettori e gli elementi di $\mathbb{K}$ sono chiamati scalari.\\
	$K=\R:$ spazio vettoriale reale\\
	$K=\mathbb{C}:$ spazio vettoriale complesso

	\ul{Osservazioni}: Sia $\mathbb{V}$ un K-spazio vettoriale:
	\begin{itemize}
		\item in $\mathbb{V}$ esiste un unico vettore nullo che denotiamo $\ul{0}$
		\item $\forall v\in V$ esiste un unico opposto che denotiamo $-v$
		\item $\forall v\in V$ si ha $0\cdot v=\ul{0}$
		\item $\forall\lambda\in K$ si ha $\lambda\cdot$ $\ul{0}=\ul{0}$
		\item Siano $\lambda\in K,v\in K:\lambda\cdot v=\ul{0}\Rightarrow$ o $v=\ul{0}$
	\end{itemize}
}

\section{\hl{16/03/22}}
\subsection*{Continuo di ieri (Def)}
\hspace*{1.3em}$+:\R^2\times\R^2\rightarrow\R^2\\
	((x_1,y_1),(x_2,y_2))\mapsto(x_1,y_1)+(x_2,y_2):=(x_1+x_2,y_1+y_2)$

$*:\R^2\times\R^2\\
	((x_1,y_1),(x_2,y_2))\mapsto(x_1,y_1)*(x_2,y_2):=(x_1\cdot x_2,y_1\cdot y_2)$

$(\R^2,+,*)$ è un campo? NO!

$\triangle:\R^2\times\R^2\\
	((a,b),(c,d))\mapsto(a,b)\triangle(c,d)=(ac-bd,ad+bc)$

Mostrare che $(\R^2,+,\triangle)$ è un campo.

\ul{Indizio}: $\R^2\leftrightarrow\mathbb{C}=\{a+ib,b\in\R,i^2=-1\}$

$(a,b)leftmapsto a+ib$

$(a,b)\mapsto a+ib$

Nota che $\mathbb{C}$ è un campo

\subsubsection*{Esempi di spazi vettoriali}
\begin{enumerate}
	\item $\mathbb{V}=\{$Vettori geometrici nello spazio$\}=\{$segmenti orientati $\vec{OP},\ P$ nello spazio$\}\longleftrightarrow\R^3$\\
	      Definiamo $+$ e $\cdot$ in modo analogo al caso dei vettori nel piano e $(V,+,\cdot)$ è uno spazio vettoriale reale.
	\item L'$n$-spazio vettoriale su $\R$ (o su $\mathbb{K}$)\\
	      $n\in\mathbb{N},n\ge1$\\
	      $\R^n=\R\times\cdots\times\R=\{(x_1,\cdots,x_n):x_i\in\R\forall i\}$\\
	      Definiamo $+:\R^n\times\R^n\to\R^n$ e $\cdot:\R^n\times\R^n\to\R^n$\\
	      $(x_1,\cdots,x_n)+(y_1,\cdots,y_n):=(x_1+y_1,\cdots,x_n+y_n)$\\
	      $\lambda\in K$ \hl{incompleto}\\
	      $\forall n\ge1,(\R^n,+,\cdot)$ è uno spazio vettoriale reale chiamato $n$-spazio vettoriale \ul{numerico} su $\R$.\\
	      Elemento \ul{neutro}: $\ul{0}=(0,\cdots,0)$\\
	      Elemento \ul{opposto} di $x=(x_1,\cdots,x_n)$ è $-x=(-x_1,\cdots,-x_n)$\\
	      \ul{Osservazione}: $n=1:\R$ è uno spazio vettoriale su $\R$ (ogni campo $\mathbb{K}$ è uno spazio vettoriale su se stesso)\\
	      In maniera analoga definiamo $+$ e $\cdot$ su\\
	      $k^n=\{(x_1,\cdots,x_n):x_i\in K\forall i\}$\\
	      $(k^n,+,\cdot)$ è uno spazio vettoriale su $\mathbb{K}$ chiamato $n$-spazio vettoriale numerico su $\mathbb{K}$.\\
	      \ul{Esempio}: $k=\mathbb{C},\mathbb{C}^n\\k=\mathbb{F}_2,\mathbb{F}^n_2$
	\item Funzioni da un insieme a un campo\\
	      Sia $\mathbb{X}$ un insieme <u>qualunque</u> e $\mathbb{K}$ un campo\\
	      $\mathbb{V}=\{$funzioni $f:\mathbb{X}\to\mathbb{K}\}$
	      \begin{itemize}
		      \item Binaria interna: $+:V\times V\to V\\(f,g)\to f+g$\\
		            dove $f+g: X\to K\\x\mapsto(f+g)(x):=f(x)+g(x)$
		      \item Binaria esterna: $\cdot:K\times V\to V\\(\lambda\cdot)$ \hl{incompleto}
	      \end{itemize}
	\item Polinomi a coefficisnti reali in una indeterminata\\
	      Sia $x$ un'indeterminata.\\
	      Un \ul{polinomio} a coefficienti reali nell'indeterminata $x$ è un'espressione formale del tipo:\\
	      $P(x)=a_nx^n+a_{n-1}x^{n-1}+\cdots+a_1x+a_0,a_i\in\R\forall i$\\
	      Se $a_n\ne0$ diremo che $n$ è il grado di $P$ e scriviamo $deg(P)=n$.\\
	      $\R[x]:=\{$polinomi a coefficienti reali nell'indeterminata $x$ (di grado arbitrario)$\}$\\
	      <u>Esempio</u>: $P(x)=3x^4+2x^3-x+5\in\R[x],\ deg(P)=4$\\
	      $+:\R[x]\times\R[x]\to\R[x]\\P(x)=a_nx^n+\cdots+a_1x+a_0\\Q(x)=b_nx^n+\cdots+b_1x+b_0\\(P+Q)(x):=(a_n+b_n)x^n+\cdots+(a_1+b_1)x+a_0+b_0$\\
	      $\cdot:\R\times\R[x]\to\R[x]\\P\in\R[x],\ P=a_nx^n+\cdots+a_1wx+a_0\\\lambda\in\R\\(\lambda\cdot P)(x):=\lambda a_nx^n+\cdots+\lambda a_1x+\lambda a_0$\\
	      $(\R[x],+,\cdot)$ è uno spazio vettoriale su $\R$.\\
	      In modo analogo si definisce $(K[x],+,\cdot)$ dove\\
	      $K[x]=\{$polinomi a coefficienti in $\mathbb{K}$ in un'indetermiinata$\}$\\
	      Molti problemi di matemadica/fisica hanno la proprietà che l'insieme delle soluzioni ha una struttura di spazio vettoriale.\\
	      \ul{Esempi}:
	      \begin{itemize}
		      \item $\begin{cases}x+2y+z=0\\y+7z=0\end{cases}$\\
		            $S=\{(x,y,z)\in\R^3:x+2y+z=0$ e $y+7z=0\}=\{(13t,-7t,t),t\in\R\}$\\
		            $S$ ha una struttuta di spazio vettoriale (\hl{cose che vedremo più avanti})
	      \end{itemize}
\end{enumerate}

\section{\hl{22/03/22}}
\subsection*{Sistemi di equazioni lineari (accenni)}
Equazione lineare in $n$ incognite:

$x_1,\cdots,x_n: a_1x_1+a_2x_2+\cdots+a_nx_n=b$

$a_i\in\R,\forall i,b\in\R$\\
Sistema di $n$ equazioni lineari in $n$ incognite:

$\begin{cases}
		a_{11}x_1+a_{12}+\cdots+a_{1n}x_n=b_1 \\
		a_{21}x_1+a_{22}+\cdots+a_{2n}x_n=b_2 \\
		\cdots                                \\
		a_{n1}x_1+a_{n2}+\cdots+a_{nn}x_n=b_n
	\end{cases}$\\

Una soluzione del sistema di sopra è un vettore $(x_1,\cdots,x_n)\in\R^n$ che verifica tutte le equazioni.

\ul{Esempi}:\vspace*{1em}

$\begin{cases}x+y+z=6\\2x-y=0\end{cases}$\vspace*{1em}

\ul{Una} soluzione di questo sistema è $(1,2,3)\in\R^3$.\vspace*{1em}

\ul{Domande}: Supponiamo di avere un sistema lineare:
\begin{enumerate}
	\item Esiste almeno una soluzione?
	\item "Quante" sono?
	\item Come si interpreta geometricamente il corrispondente insieme di soluzioni?
\end{enumerate}
\subsection*{\ul{Matrici}}
\begin{itemize}
	\item Un esempio di spazio vettoriale
	\item Uno strumento conciso per rappresentare oggetti *parola incomprensibile*, tra cui molti dell'algebra lineare
\end{itemize}
Sia $\mathbb{K}$ un campo.
Siano $m,n\ge1$ due numeri

\Def{Una matrice $m\times n$ a elementi in $\mathbb{K}$ è una tabella rettangolare di $m\cdot n$ elementi di $\mathbb{K}$, disposti su $m$ righe e $n$ colonne}{}

\ul{Notazione}:

$A=\begin{bmatrix}
		a_{11} & a_{12} & \cdots & a_{ij} & \cdots & a_{1n} \\
		a_{21} & a_{22} & \cdots & a_{ij} & \cdots & a_{2n} \\
		\vdots                                              \\
		a_{i1} & a_{i2} & \cdots & a_{ij} & \cdots & a_{in} \\
		\vdots                                              \\
		a_{n1} & a_{n2} & \cdots & a_{nj} & \cdots & a_{nn} \\
	\end{bmatrix}$

$(a_{ij})_{1\le i\le n\ \wedge\ 1\le j\le n}:i$ è la riga e $j$ è la colonna, come in c++ (o qualunque linguaggio "normale")

Ciascuno degli elementi $a_{ij}$ è detto entrata (o coefficiente) delle matrici

\subsubsection*{\ul{Un po' di terminologia}}
\begin{itemize}
	\item Se $n=m$, una matrice $n\times n$ si dice matrice \ul{quadrata} di ordine $n$,\\
	      ha due diagonali (solo se è quadrata eh)
	\item Una matrice $1\times n$ è chiamata vettore riga.\\
	      Una matrice $n\times1$ è chiamata vettore colonna.
\end{itemize}
\ul{Notazione}:

$M_{m,n}(K)=\{$matrici $n\times m$ a coefficienti in $K\}$

M$_{n}(K):=\{$matrici quadrete di ordine $n\}$

\Def{Siano $A=(a_{ij}),\ B=(b_{ij})\in M_{m,n}(K)$.}{
Diciamo che $A=B$ se $a_{ij}=b_{ij}\forall 1\le i \le n,\ 1\le j \le n$\\
Definiamo due operazioni su $M_{m,n}$:
\begin{itemize}
	\item \ul{Somma di matrici}\\
	      $+: M_{m,n}\times M_{m,n}\rightarrow M_{m,n}(K)\\\hspace*{4.7em}(A,B)\longmapsto A+B$\\

	      $A=\begin{bmatrix}\\
			      a_{11} & \cdots & a_{1n} \\
			      \vdots                   \\
			      a_{m1} & \cdots & a_{mn} \\
		      \end{bmatrix}=(a_{ij})\in M_{m,n}(K)$\\

	      $B=\begin{bmatrix}\\
			      b_{11} & \cdots & b_{1n} \\
			      \vdots                   \\
			      b_{m1} & \cdots & b_{mn} \\
		      \end{bmatrix}=(b_{ij})\in M_{m,n}(K)$\\

	      $A+B:=\begin{bmatrix}\\
			      a_{11}+b_{11} & \cdots & a_{1n}+b_{1n} \\
			      \vdots                                 \\
			      a_{m1}+b_{m1} & \cdots & a_{mn}+b_{mn} \\
		      \end{bmatrix}=(a_{ij}+b_{ij})$ \hl{incompleto}, probabilmente $\in M_{m,n}(K)$
	\item \ul{Moltiplicazione per scalari}

	      $\cdot:K\times M_{m,n}(K)\rightarrow M_{m,n}(K)\\\hspace*{4.6em}(\lambda,A)\longmapsto\lambda\cdot A$

	      $A=(a_{ij})\in M_{m,n}(K)\\\lambda\in K$\\
	      $\lambda\cdot A=(\lambda a_{ij})\in M_{m,n}(K)$\\

	      \ul{Proprietà}:
	      \begin{enumerate}
		      \item $+$ è commutativa: $A+B=B+A$
		      \item $+$ è associativa: $(A+B)+C=A+(B+C)$
		      \item Elemento neutro rispetto a $+$:\\
		            $0_{m,n}=\begin{bmatrix}0&\cdots&0\quad\vdots\quad0&\cdots&0\end{bmatrix}$
		      \item Elemento opposto rispetto a $+$:\\
		            $A=(a_{ij})\Rightarrow-A=(-a_{ij})$
		      \item $\lambda\cdot(A+B)=\lambda\cdot A + \lambda\cdot B$
		      \item $(\lambda+\mu)\cdot A=\lambda\cdot A+\mu\cdot A$
		      \item $(\lambda\mu)\cdot A=\lambda\cdot(\mu\cdot A)$
		      \item $1\cdot A = A\forall m,n\ge1\text{ interi, }(M_{m,n}(K),+,\cdot)\text{ è uno spazio vettoriale su }\mathbb{K}$
	      \end{enumerate}
	\item \ul{Prodotto di matrici} (prodotto riga per colonna):

	      $(a_1\cdots\ a_n)\in M_{1,n}(K):$ vettore riga

	      $(b_1\cdots\ b_n)\in M_{n,1}(K):$ vettore colonna

	      $(a_1\cdots\ a_n)\cdot(b_1\cdots\ b_n):=a_1b_1+\cdots+a_nb_n=\sum^n_{k=1}{a_kb_k}$\\

	      Generalizziamo al prodotto di due matrici:\\

	      $\begin{bmatrix}
			      a_{11} & \cdots & a_{1n} \\
			      a_{m1} & \cdots & a_{mn}
		      \end{bmatrix}
		      \cdot
		      \begin{bmatrix}
			      b_{11} & \cdots & b_{1p} \\
			      b_{q1} & \cdots & b_{qp}
		      \end{bmatrix}=
		      \begin{bmatrix}
			      c_{ij}
		      \end{bmatrix}$

	      $c_{ij}=$prodotto della $i$-esima riga di $A$ e della $j$-esima colonna di $B$

	      Per definire il prodotto abbiamo bisogno che $n=q$

	      $\forall\ 1\ge i\ge n,\ \forall\ 1\ge j\ge p$ definiamo

	      $c_{ij}=(a_{i1}\cdots a_{in})\cdot(b_{1j}\cdots b_{nj})=a_{i1}b_{1j}+\cdots+a_{in}b_{nj}=\sum^n_{n=1}{a_{ik}b_{kj}}$

	      Più formalmente, il prodotto di due matrici è una funzione:

	      $M_{m,n}(K)\times M_{n,p}(K)\rightarrow M_{m,p}(K)\\\hspace*{6.5em}(A,B)\mapsto C=AB$

	      $A=(a_{ij}): 1\le i\le m,\ 1\le j\le n$

	      $B=(b_{ij}): 1\le i\le n,\ 1\le j\le p$

	      $C=AB=(c_{ij}): 1\le i\le m,\ 1\le j\le p$

	      \ul{NOTA BENE}: il prodotto è definito se e solo se il numero di colonne di $A$ è uguale al numero di righe di $B$.

	      \ul{Caso particolare}

	      $n=p=m:$ otteniamo un'operazione binaria interna su $M_n(K)$
\end{itemize}
}
\end{document}