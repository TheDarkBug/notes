\documentclass{article}
\usepackage{hyperref}
\usepackage{ocgx2}
\usepackage{color,soul}
\usepackage{amsfonts}
\usepackage{parskip}

\begin{document}
% 
\newcommand{\Def}[2]{\paragraph{\underline{Def}:}#1\\\hspace*{3em}\begin{minipage}{.8\textwidth}#2\end{minipage}}
\newcommand{\R}{\mathbb{R}}
\pagenumbering{gobble}
\title{Appunti del corso di algebra e geometria}
\author{Adriano Oliviero}
\section*{\hl{Introduzione}}
Questi appunti sono scritti seguendo le lezioni della prof.ssa \href{https://aiezzi.it/}{Anna Iezzi} e sono distribuiti sotto la \hyperref[sec:License]{licenza}. Il repository originario per questi (ed altri appunti) è su \href{https://github.com/TheDarkBug/notes}{github}.

\section*{\hl{License}}
\label{sec:License}
TO BE FILLED
\section*{\hl{08/03/22}}
\subsection*{Richiami di logica}
Proposizione logica: \textbf{V}/\textbf{F}, \textbf{1}/\textbf{0} valore di verità
P = "Napoli è in Campania": \textbf{V}
P(n) = "n è pari"
P(2): \textbf{V}
P(3): \textbf{F}
\subsection*{Connettivi logici}
\begin{itemize}
	\item Negazione: $\neg$, "non"
	\item Congiunzione: $\wedge$, "e"
	\item Disgiunzione: $\vee$, "o"
	\item Implicazione: $\Rightarrow$, "se $\cdots$ allora $\cdots$"
	\item Doppia implicazione: $\Leftrightarrow$, "se e solo se"
\end{itemize}
\subsection*{Tavole di verità}
\begin{displaymath}
\begin{array}{|c|c|c|c|c|c|c|c|c|c|c|}
\hline
P&Q&\neg P&P\wedge Q&P\vee Q&P\Rightarrow Q&P\Leftrightarrow Q\\
\hline\hline
1&1&0&1&1&1&1\\
\hline
1&0&0&0&1&0&0\\
\hline
0&1&1&0&1&1&0\\
\hline
0&0&1&0&0&1&1\\
\hline
\end{array}
\end{displaymath}
\subsection*{Quantificatori}
\begin{itemize}
	\item $\forall$: "Per ogni"
	\item $\exists$: "Esiste almeno uno"
	\item $\exists!$: "Esiste ed è unico"
\end{itemize}

\underline{Esempi}:
\begin{enumerate}
	\item "$\forall n$ naturale, $n$ è pari": F\\
		  "$\exists n$ naturale: ($\leftarrow$ tale che) $n$ è pari": V
	\item $P(x)=\{x$ è uno studente in aula A3-T2$\}$\\
		  $Q(x)=\{x$ è iscritto ad un corso di ingegneria$\}$\\
		  $\forall x, P(x)\Rightarrow Q(x)$
	\item $P(n)=\{n$ è un numero pari$\}$\\
		  $Q(n)=\{n$ è divisibile per $4\}$\\
		  $\forall n$ naturale, $Q(n)\Rightarrow P(n)$
\end{enumerate}

\subsection*{Teoria degli insiemi}
Un \textbf{insieme} è una collezione di oggetti detti \textbf{elementi} dell'insieme.\\
Convenzionalmente gli insiemi si denotano con lettere maiuscole e gli elementi con lettere minuscole.
\subsubsection*{Descrizione di un insieme}
\begin{enumerate}
	\item Per elencazione (se l'insieme ha un numero finito di elementi):\\
		  $\mathbb{A}=\{0,2,4,6,8,10\}$\\
		  $4\in\mathbb{A}$\\
		  $5\notin\mathbb{A}$
	\item Per proprietà caratteristica:\\
		  $\mathbb{A}=\{n:n$ è un numero pari: $0\le n\le10\}$
	\item Diagramma di \textit{Eulero-Venn} (ancora una volta se l'insieme ha un numero finito di elementi):
\end{enumerate}
\subsubsection*{Alcune informazioni sugli insiemi}
\begin{itemize}
	\item La \textbf{cardinalità} di un insieme è il numero di elementi che un insieme contiene e si denota: $\mathbb{A}$ ha cardinalità $|\mathbb{A}|$
	\item Insieme vuoto: insieme che non contiene nessun elemento, si denota con $\emptyset$, $\{\}$ e ha cardinalità $|\emptyset|=0$
	\item I principali insiemi numerici:\\
		  $\mathbb{N}=\{0,1,2,3,\cdots\}$: numeri naturali\\
		  $\mathbb{Z}=\{\cdots,-3,-2,-1,0,1,2,3,\cdots\}$: numeri naturali\\
		  $\mathbb{Q}=\{\frac{a}{b}:a,b$ e $\mathbb{Z},b\not=0\}$\\
		  $\R$: numeri reali\\
		  $\mathbb{C}$: numeri complessi
\end{itemize}

\subsubsection*{Operazioni tra gli insiemi}
\begin{enumerate}
	\item Inclusione $\leftrightarrow\subset$:\\
		  $\mathbb{A}\subseteq\mathbb{B}\quad(\mathbb{A}$ è contenuto in $\mathbb{B})$\\
		  $\mathbb{B}\supseteq\mathbb{A}$ $(\mathbb{B}$ contiene $\mathbb{A})$\\
		  \underline{Esempio}:\\
		  $\mathbb{N}\subseteq\mathbb{Z}\subseteq\mathbb{Q}\subseteq\R\subseteq\mathbb{C}$\\
		  \underline{Proprietà}:
		  \begin{itemize}
			  \item $\mathbb{A}=\mathbb{B}\Leftrightarrow(x\in\mathbb{A}\Leftrightarrow x\in\mathbb{B})\Leftrightarrow(\mathbb{A}\subseteq\mathbb{B}\wedge\mathbb{B}\subseteq\mathbb{A})$
		  \end{itemize}
	\item Intersezione $\leftrightarrow\wedge$:\\
		  $\mathbb{A}\cap\mathbb{B}=\{x:x\in\mathbb{A}\wedge x\in\mathbb{B}\}$\\
		  \underline{Proprietà}:
		  \begin{itemize}
			  \item $\mathbb{A}\cap\emptyset=\emptyset$
			  \item $\mathbb{A}\cap\mathbb{B}=\mathbb{B}\cap\mathbb{A}$
			  \item $\mathbb{A}\subseteq\mathbb{B}\Rightarrow\mathbb{A}\cap\mathbb{B}=\mathbb{A}$
		  \end{itemize}
	\item Unione $\leftrightarrow\vee$:\\
		  $\mathbb{A}\cup\mathbb{B}=\{x:x\in\mathbb{A}\vee x\in\mathbb{B}\}$\\
		  \underline{Proprietà}:
		  \begin{itemize}
			  \item $\mathbb{A}\cup\emptyset=\mathbb{A}$
			  \item $\mathbb{A}\cup\mathbb{B}=\mathbb{B}\cup\mathbb{A}$
			  \item $\mathbb{A}\subseteq\mathbb{B}\Rightarrow\mathbb{A}\cup\mathbb{B}=\mathbb{B}$
			  \item $\mathbb{A}\subseteq\mathbb{A}\cup\mathbb{B},\mathbb{B}\subseteq\mathbb{A}\cup\mathbb{B}$
		  \end{itemize}
	\item Differenza $\leftrightarrow\backslash$:\\
		  $\mathbb{B}\backslash\mathbb{A}=\{x:x\in\mathbb{B}\wedge x\not\in\mathbb{A}\}$
	\item Prodotto cartesiano $\leftrightarrow\times$:\\
		  $\mathbb{A}\times\mathbb{B}:=\{(a,b):a\in\mathbb{A},b\in\mathbb{B}\}$ con $(a,b)$ coppie ordinate\\
		  \underline{Proprietà}:
		  \begin{itemize}
			  \item $|\mathbb{A}\times\mathbb{B}|=|\mathbb{A}|\cdot|\mathbb{B}|$
			  \item $\mathbb{A}\times\emptyset=\emptyset=\emptyset\times\mathbb{A}$
		  \end{itemize}
\end{enumerate}
\subsection*{Funzioni}
\Def{Siano $\mathbb{A},\mathbb{B}$ due insiemi:}
{Una funzione $f:\mathbb{A}\rightarrow\mathbb{B}$ è una legge che associa ad ogni elemento di $\mathbb{A}$, uno ed un solo elemento di $\mathbb{B}$.\\
$f:\mathbb{A}\rightarrow\mathbb{B}$\\
\hspace*{1.6em}$x\rightarrow y:\quad x$ è la controimmagine di $y$}
\Def{Sia $f:\mathbb{A}\rightarrow\mathbb{B}$ una funzione e sia $\mathbb{X}\subseteq\mathbb{A}$:}
	{$f(x):=\{f(x):x\in\mathbb{X}\}$ è l'insieme di $\mathbb{X}$ tramite $f$2
	$Im(f):=f(\mathbb{A})$ è l'insieme della funzione\\
	$Im(f)$ non è necessariamente $=Dom(f)$\\
	$Im(f)=Dom(f)\Rightarrow f$ è suriettiva\\
	$f(1)=f(2)=a\Rightarrow f$ non è iniettiva}
\end{document}

